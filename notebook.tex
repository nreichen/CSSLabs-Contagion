
% Default to the notebook output style

    


% Inherit from the specified cell style.




    
\documentclass[11pt]{article}

    
    
    \usepackage[T1]{fontenc}
    % Nicer default font (+ math font) than Computer Modern for most use cases
    \usepackage{mathpazo}

    % Basic figure setup, for now with no caption control since it's done
    % automatically by Pandoc (which extracts ![](path) syntax from Markdown).
    \usepackage{graphicx}
    % We will generate all images so they have a width \maxwidth. This means
    % that they will get their normal width if they fit onto the page, but
    % are scaled down if they would overflow the margins.
    \makeatletter
    \def\maxwidth{\ifdim\Gin@nat@width>\linewidth\linewidth
    \else\Gin@nat@width\fi}
    \makeatother
    \let\Oldincludegraphics\includegraphics
    % Set max figure width to be 80% of text width, for now hardcoded.
    \renewcommand{\includegraphics}[1]{\Oldincludegraphics[width=.8\maxwidth]{#1}}
    % Ensure that by default, figures have no caption (until we provide a
    % proper Figure object with a Caption API and a way to capture that
    % in the conversion process - todo).
    \usepackage{caption}
    \DeclareCaptionLabelFormat{nolabel}{}
    \captionsetup{labelformat=nolabel}

    \usepackage{adjustbox} % Used to constrain images to a maximum size 
    \usepackage{xcolor} % Allow colors to be defined
    \usepackage{enumerate} % Needed for markdown enumerations to work
    \usepackage{geometry} % Used to adjust the document margins
    \usepackage{amsmath} % Equations
    \usepackage{amssymb} % Equations
    \usepackage{textcomp} % defines textquotesingle
    % Hack from http://tex.stackexchange.com/a/47451/13684:
    \AtBeginDocument{%
        \def\PYZsq{\textquotesingle}% Upright quotes in Pygmentized code
    }
    \usepackage{upquote} % Upright quotes for verbatim code
    \usepackage{eurosym} % defines \euro
    \usepackage[mathletters]{ucs} % Extended unicode (utf-8) support
    \usepackage[utf8x]{inputenc} % Allow utf-8 characters in the tex document
    \usepackage{fancyvrb} % verbatim replacement that allows latex
    \usepackage{grffile} % extends the file name processing of package graphics 
                         % to support a larger range 
    % The hyperref package gives us a pdf with properly built
    % internal navigation ('pdf bookmarks' for the table of contents,
    % internal cross-reference links, web links for URLs, etc.)
    \usepackage{hyperref}
    \usepackage{longtable} % longtable support required by pandoc >1.10
    \usepackage{booktabs}  % table support for pandoc > 1.12.2
    \usepackage[inline]{enumitem} % IRkernel/repr support (it uses the enumerate* environment)
    \usepackage[normalem]{ulem} % ulem is needed to support strikethroughs (\sout)
                                % normalem makes italics be italics, not underlines
    

    
    
    % Colors for the hyperref package
    \definecolor{urlcolor}{rgb}{0,.145,.698}
    \definecolor{linkcolor}{rgb}{.71,0.21,0.01}
    \definecolor{citecolor}{rgb}{.12,.54,.11}

    % ANSI colors
    \definecolor{ansi-black}{HTML}{3E424D}
    \definecolor{ansi-black-intense}{HTML}{282C36}
    \definecolor{ansi-red}{HTML}{E75C58}
    \definecolor{ansi-red-intense}{HTML}{B22B31}
    \definecolor{ansi-green}{HTML}{00A250}
    \definecolor{ansi-green-intense}{HTML}{007427}
    \definecolor{ansi-yellow}{HTML}{DDB62B}
    \definecolor{ansi-yellow-intense}{HTML}{B27D12}
    \definecolor{ansi-blue}{HTML}{208FFB}
    \definecolor{ansi-blue-intense}{HTML}{0065CA}
    \definecolor{ansi-magenta}{HTML}{D160C4}
    \definecolor{ansi-magenta-intense}{HTML}{A03196}
    \definecolor{ansi-cyan}{HTML}{60C6C8}
    \definecolor{ansi-cyan-intense}{HTML}{258F8F}
    \definecolor{ansi-white}{HTML}{C5C1B4}
    \definecolor{ansi-white-intense}{HTML}{A1A6B2}

    % commands and environments needed by pandoc snippets
    % extracted from the output of `pandoc -s`
    \providecommand{\tightlist}{%
      \setlength{\itemsep}{0pt}\setlength{\parskip}{0pt}}
    \DefineVerbatimEnvironment{Highlighting}{Verbatim}{commandchars=\\\{\}}
    % Add ',fontsize=\small' for more characters per line
    \newenvironment{Shaded}{}{}
    \newcommand{\KeywordTok}[1]{\textcolor[rgb]{0.00,0.44,0.13}{\textbf{{#1}}}}
    \newcommand{\DataTypeTok}[1]{\textcolor[rgb]{0.56,0.13,0.00}{{#1}}}
    \newcommand{\DecValTok}[1]{\textcolor[rgb]{0.25,0.63,0.44}{{#1}}}
    \newcommand{\BaseNTok}[1]{\textcolor[rgb]{0.25,0.63,0.44}{{#1}}}
    \newcommand{\FloatTok}[1]{\textcolor[rgb]{0.25,0.63,0.44}{{#1}}}
    \newcommand{\CharTok}[1]{\textcolor[rgb]{0.25,0.44,0.63}{{#1}}}
    \newcommand{\StringTok}[1]{\textcolor[rgb]{0.25,0.44,0.63}{{#1}}}
    \newcommand{\CommentTok}[1]{\textcolor[rgb]{0.38,0.63,0.69}{\textit{{#1}}}}
    \newcommand{\OtherTok}[1]{\textcolor[rgb]{0.00,0.44,0.13}{{#1}}}
    \newcommand{\AlertTok}[1]{\textcolor[rgb]{1.00,0.00,0.00}{\textbf{{#1}}}}
    \newcommand{\FunctionTok}[1]{\textcolor[rgb]{0.02,0.16,0.49}{{#1}}}
    \newcommand{\RegionMarkerTok}[1]{{#1}}
    \newcommand{\ErrorTok}[1]{\textcolor[rgb]{1.00,0.00,0.00}{\textbf{{#1}}}}
    \newcommand{\NormalTok}[1]{{#1}}
    
    % Additional commands for more recent versions of Pandoc
    \newcommand{\ConstantTok}[1]{\textcolor[rgb]{0.53,0.00,0.00}{{#1}}}
    \newcommand{\SpecialCharTok}[1]{\textcolor[rgb]{0.25,0.44,0.63}{{#1}}}
    \newcommand{\VerbatimStringTok}[1]{\textcolor[rgb]{0.25,0.44,0.63}{{#1}}}
    \newcommand{\SpecialStringTok}[1]{\textcolor[rgb]{0.73,0.40,0.53}{{#1}}}
    \newcommand{\ImportTok}[1]{{#1}}
    \newcommand{\DocumentationTok}[1]{\textcolor[rgb]{0.73,0.13,0.13}{\textit{{#1}}}}
    \newcommand{\AnnotationTok}[1]{\textcolor[rgb]{0.38,0.63,0.69}{\textbf{\textit{{#1}}}}}
    \newcommand{\CommentVarTok}[1]{\textcolor[rgb]{0.38,0.63,0.69}{\textbf{\textit{{#1}}}}}
    \newcommand{\VariableTok}[1]{\textcolor[rgb]{0.10,0.09,0.49}{{#1}}}
    \newcommand{\ControlFlowTok}[1]{\textcolor[rgb]{0.00,0.44,0.13}{\textbf{{#1}}}}
    \newcommand{\OperatorTok}[1]{\textcolor[rgb]{0.40,0.40,0.40}{{#1}}}
    \newcommand{\BuiltInTok}[1]{{#1}}
    \newcommand{\ExtensionTok}[1]{{#1}}
    \newcommand{\PreprocessorTok}[1]{\textcolor[rgb]{0.74,0.48,0.00}{{#1}}}
    \newcommand{\AttributeTok}[1]{\textcolor[rgb]{0.49,0.56,0.16}{{#1}}}
    \newcommand{\InformationTok}[1]{\textcolor[rgb]{0.38,0.63,0.69}{\textbf{\textit{{#1}}}}}
    \newcommand{\WarningTok}[1]{\textcolor[rgb]{0.38,0.63,0.69}{\textbf{\textit{{#1}}}}}
    
    
    % Define a nice break command that doesn't care if a line doesn't already
    % exist.
    \def\br{\hspace*{\fill} \\* }
    % Math Jax compatability definitions
    \def\gt{>}
    \def\lt{<}
    % Document parameters
    \title{lab\_1\_metoo}
    
    
    

    % Pygments definitions
    
\makeatletter
\def\PY@reset{\let\PY@it=\relax \let\PY@bf=\relax%
    \let\PY@ul=\relax \let\PY@tc=\relax%
    \let\PY@bc=\relax \let\PY@ff=\relax}
\def\PY@tok#1{\csname PY@tok@#1\endcsname}
\def\PY@toks#1+{\ifx\relax#1\empty\else%
    \PY@tok{#1}\expandafter\PY@toks\fi}
\def\PY@do#1{\PY@bc{\PY@tc{\PY@ul{%
    \PY@it{\PY@bf{\PY@ff{#1}}}}}}}
\def\PY#1#2{\PY@reset\PY@toks#1+\relax+\PY@do{#2}}

\expandafter\def\csname PY@tok@w\endcsname{\def\PY@tc##1{\textcolor[rgb]{0.73,0.73,0.73}{##1}}}
\expandafter\def\csname PY@tok@c\endcsname{\let\PY@it=\textit\def\PY@tc##1{\textcolor[rgb]{0.25,0.50,0.50}{##1}}}
\expandafter\def\csname PY@tok@cp\endcsname{\def\PY@tc##1{\textcolor[rgb]{0.74,0.48,0.00}{##1}}}
\expandafter\def\csname PY@tok@k\endcsname{\let\PY@bf=\textbf\def\PY@tc##1{\textcolor[rgb]{0.00,0.50,0.00}{##1}}}
\expandafter\def\csname PY@tok@kp\endcsname{\def\PY@tc##1{\textcolor[rgb]{0.00,0.50,0.00}{##1}}}
\expandafter\def\csname PY@tok@kt\endcsname{\def\PY@tc##1{\textcolor[rgb]{0.69,0.00,0.25}{##1}}}
\expandafter\def\csname PY@tok@o\endcsname{\def\PY@tc##1{\textcolor[rgb]{0.40,0.40,0.40}{##1}}}
\expandafter\def\csname PY@tok@ow\endcsname{\let\PY@bf=\textbf\def\PY@tc##1{\textcolor[rgb]{0.67,0.13,1.00}{##1}}}
\expandafter\def\csname PY@tok@nb\endcsname{\def\PY@tc##1{\textcolor[rgb]{0.00,0.50,0.00}{##1}}}
\expandafter\def\csname PY@tok@nf\endcsname{\def\PY@tc##1{\textcolor[rgb]{0.00,0.00,1.00}{##1}}}
\expandafter\def\csname PY@tok@nc\endcsname{\let\PY@bf=\textbf\def\PY@tc##1{\textcolor[rgb]{0.00,0.00,1.00}{##1}}}
\expandafter\def\csname PY@tok@nn\endcsname{\let\PY@bf=\textbf\def\PY@tc##1{\textcolor[rgb]{0.00,0.00,1.00}{##1}}}
\expandafter\def\csname PY@tok@ne\endcsname{\let\PY@bf=\textbf\def\PY@tc##1{\textcolor[rgb]{0.82,0.25,0.23}{##1}}}
\expandafter\def\csname PY@tok@nv\endcsname{\def\PY@tc##1{\textcolor[rgb]{0.10,0.09,0.49}{##1}}}
\expandafter\def\csname PY@tok@no\endcsname{\def\PY@tc##1{\textcolor[rgb]{0.53,0.00,0.00}{##1}}}
\expandafter\def\csname PY@tok@nl\endcsname{\def\PY@tc##1{\textcolor[rgb]{0.63,0.63,0.00}{##1}}}
\expandafter\def\csname PY@tok@ni\endcsname{\let\PY@bf=\textbf\def\PY@tc##1{\textcolor[rgb]{0.60,0.60,0.60}{##1}}}
\expandafter\def\csname PY@tok@na\endcsname{\def\PY@tc##1{\textcolor[rgb]{0.49,0.56,0.16}{##1}}}
\expandafter\def\csname PY@tok@nt\endcsname{\let\PY@bf=\textbf\def\PY@tc##1{\textcolor[rgb]{0.00,0.50,0.00}{##1}}}
\expandafter\def\csname PY@tok@nd\endcsname{\def\PY@tc##1{\textcolor[rgb]{0.67,0.13,1.00}{##1}}}
\expandafter\def\csname PY@tok@s\endcsname{\def\PY@tc##1{\textcolor[rgb]{0.73,0.13,0.13}{##1}}}
\expandafter\def\csname PY@tok@sd\endcsname{\let\PY@it=\textit\def\PY@tc##1{\textcolor[rgb]{0.73,0.13,0.13}{##1}}}
\expandafter\def\csname PY@tok@si\endcsname{\let\PY@bf=\textbf\def\PY@tc##1{\textcolor[rgb]{0.73,0.40,0.53}{##1}}}
\expandafter\def\csname PY@tok@se\endcsname{\let\PY@bf=\textbf\def\PY@tc##1{\textcolor[rgb]{0.73,0.40,0.13}{##1}}}
\expandafter\def\csname PY@tok@sr\endcsname{\def\PY@tc##1{\textcolor[rgb]{0.73,0.40,0.53}{##1}}}
\expandafter\def\csname PY@tok@ss\endcsname{\def\PY@tc##1{\textcolor[rgb]{0.10,0.09,0.49}{##1}}}
\expandafter\def\csname PY@tok@sx\endcsname{\def\PY@tc##1{\textcolor[rgb]{0.00,0.50,0.00}{##1}}}
\expandafter\def\csname PY@tok@m\endcsname{\def\PY@tc##1{\textcolor[rgb]{0.40,0.40,0.40}{##1}}}
\expandafter\def\csname PY@tok@gh\endcsname{\let\PY@bf=\textbf\def\PY@tc##1{\textcolor[rgb]{0.00,0.00,0.50}{##1}}}
\expandafter\def\csname PY@tok@gu\endcsname{\let\PY@bf=\textbf\def\PY@tc##1{\textcolor[rgb]{0.50,0.00,0.50}{##1}}}
\expandafter\def\csname PY@tok@gd\endcsname{\def\PY@tc##1{\textcolor[rgb]{0.63,0.00,0.00}{##1}}}
\expandafter\def\csname PY@tok@gi\endcsname{\def\PY@tc##1{\textcolor[rgb]{0.00,0.63,0.00}{##1}}}
\expandafter\def\csname PY@tok@gr\endcsname{\def\PY@tc##1{\textcolor[rgb]{1.00,0.00,0.00}{##1}}}
\expandafter\def\csname PY@tok@ge\endcsname{\let\PY@it=\textit}
\expandafter\def\csname PY@tok@gs\endcsname{\let\PY@bf=\textbf}
\expandafter\def\csname PY@tok@gp\endcsname{\let\PY@bf=\textbf\def\PY@tc##1{\textcolor[rgb]{0.00,0.00,0.50}{##1}}}
\expandafter\def\csname PY@tok@go\endcsname{\def\PY@tc##1{\textcolor[rgb]{0.53,0.53,0.53}{##1}}}
\expandafter\def\csname PY@tok@gt\endcsname{\def\PY@tc##1{\textcolor[rgb]{0.00,0.27,0.87}{##1}}}
\expandafter\def\csname PY@tok@err\endcsname{\def\PY@bc##1{\setlength{\fboxsep}{0pt}\fcolorbox[rgb]{1.00,0.00,0.00}{1,1,1}{\strut ##1}}}
\expandafter\def\csname PY@tok@kc\endcsname{\let\PY@bf=\textbf\def\PY@tc##1{\textcolor[rgb]{0.00,0.50,0.00}{##1}}}
\expandafter\def\csname PY@tok@kd\endcsname{\let\PY@bf=\textbf\def\PY@tc##1{\textcolor[rgb]{0.00,0.50,0.00}{##1}}}
\expandafter\def\csname PY@tok@kn\endcsname{\let\PY@bf=\textbf\def\PY@tc##1{\textcolor[rgb]{0.00,0.50,0.00}{##1}}}
\expandafter\def\csname PY@tok@kr\endcsname{\let\PY@bf=\textbf\def\PY@tc##1{\textcolor[rgb]{0.00,0.50,0.00}{##1}}}
\expandafter\def\csname PY@tok@bp\endcsname{\def\PY@tc##1{\textcolor[rgb]{0.00,0.50,0.00}{##1}}}
\expandafter\def\csname PY@tok@fm\endcsname{\def\PY@tc##1{\textcolor[rgb]{0.00,0.00,1.00}{##1}}}
\expandafter\def\csname PY@tok@vc\endcsname{\def\PY@tc##1{\textcolor[rgb]{0.10,0.09,0.49}{##1}}}
\expandafter\def\csname PY@tok@vg\endcsname{\def\PY@tc##1{\textcolor[rgb]{0.10,0.09,0.49}{##1}}}
\expandafter\def\csname PY@tok@vi\endcsname{\def\PY@tc##1{\textcolor[rgb]{0.10,0.09,0.49}{##1}}}
\expandafter\def\csname PY@tok@vm\endcsname{\def\PY@tc##1{\textcolor[rgb]{0.10,0.09,0.49}{##1}}}
\expandafter\def\csname PY@tok@sa\endcsname{\def\PY@tc##1{\textcolor[rgb]{0.73,0.13,0.13}{##1}}}
\expandafter\def\csname PY@tok@sb\endcsname{\def\PY@tc##1{\textcolor[rgb]{0.73,0.13,0.13}{##1}}}
\expandafter\def\csname PY@tok@sc\endcsname{\def\PY@tc##1{\textcolor[rgb]{0.73,0.13,0.13}{##1}}}
\expandafter\def\csname PY@tok@dl\endcsname{\def\PY@tc##1{\textcolor[rgb]{0.73,0.13,0.13}{##1}}}
\expandafter\def\csname PY@tok@s2\endcsname{\def\PY@tc##1{\textcolor[rgb]{0.73,0.13,0.13}{##1}}}
\expandafter\def\csname PY@tok@sh\endcsname{\def\PY@tc##1{\textcolor[rgb]{0.73,0.13,0.13}{##1}}}
\expandafter\def\csname PY@tok@s1\endcsname{\def\PY@tc##1{\textcolor[rgb]{0.73,0.13,0.13}{##1}}}
\expandafter\def\csname PY@tok@mb\endcsname{\def\PY@tc##1{\textcolor[rgb]{0.40,0.40,0.40}{##1}}}
\expandafter\def\csname PY@tok@mf\endcsname{\def\PY@tc##1{\textcolor[rgb]{0.40,0.40,0.40}{##1}}}
\expandafter\def\csname PY@tok@mh\endcsname{\def\PY@tc##1{\textcolor[rgb]{0.40,0.40,0.40}{##1}}}
\expandafter\def\csname PY@tok@mi\endcsname{\def\PY@tc##1{\textcolor[rgb]{0.40,0.40,0.40}{##1}}}
\expandafter\def\csname PY@tok@il\endcsname{\def\PY@tc##1{\textcolor[rgb]{0.40,0.40,0.40}{##1}}}
\expandafter\def\csname PY@tok@mo\endcsname{\def\PY@tc##1{\textcolor[rgb]{0.40,0.40,0.40}{##1}}}
\expandafter\def\csname PY@tok@ch\endcsname{\let\PY@it=\textit\def\PY@tc##1{\textcolor[rgb]{0.25,0.50,0.50}{##1}}}
\expandafter\def\csname PY@tok@cm\endcsname{\let\PY@it=\textit\def\PY@tc##1{\textcolor[rgb]{0.25,0.50,0.50}{##1}}}
\expandafter\def\csname PY@tok@cpf\endcsname{\let\PY@it=\textit\def\PY@tc##1{\textcolor[rgb]{0.25,0.50,0.50}{##1}}}
\expandafter\def\csname PY@tok@c1\endcsname{\let\PY@it=\textit\def\PY@tc##1{\textcolor[rgb]{0.25,0.50,0.50}{##1}}}
\expandafter\def\csname PY@tok@cs\endcsname{\let\PY@it=\textit\def\PY@tc##1{\textcolor[rgb]{0.25,0.50,0.50}{##1}}}

\def\PYZbs{\char`\\}
\def\PYZus{\char`\_}
\def\PYZob{\char`\{}
\def\PYZcb{\char`\}}
\def\PYZca{\char`\^}
\def\PYZam{\char`\&}
\def\PYZlt{\char`\<}
\def\PYZgt{\char`\>}
\def\PYZsh{\char`\#}
\def\PYZpc{\char`\%}
\def\PYZdl{\char`\$}
\def\PYZhy{\char`\-}
\def\PYZsq{\char`\'}
\def\PYZdq{\char`\"}
\def\PYZti{\char`\~}
% for compatibility with earlier versions
\def\PYZat{@}
\def\PYZlb{[}
\def\PYZrb{]}
\makeatother


    % Exact colors from NB
    \definecolor{incolor}{rgb}{0.0, 0.0, 0.5}
    \definecolor{outcolor}{rgb}{0.545, 0.0, 0.0}



    
    % Prevent overflowing lines due to hard-to-break entities
    \sloppy 
    % Setup hyperref package
    \hypersetup{
      breaklinks=true,  % so long urls are correctly broken across lines
      colorlinks=true,
      urlcolor=urlcolor,
      linkcolor=linkcolor,
      citecolor=citecolor,
      }
    % Slightly bigger margins than the latex defaults
    
    \geometry{verbose,tmargin=1in,bmargin=1in,lmargin=1in,rmargin=1in}
    
    

    \begin{document}
    
    
    \maketitle
    
    

    
    \hypertarget{complicating-social-contagion}{%
\section{Complicating Social
Contagion}\label{complicating-social-contagion}}

    \hypertarget{section-1-background}{%
\subsection{Section 1: Background}\label{section-1-background}}

\hypertarget{learning-objectives}{%
\subsubsection{1.1 Learning Objectives}\label{learning-objectives}}

\begin{itemize}
\tightlist
\item
  Explore different aspects of social influence that affect social
  contagion processes.
\end{itemize}

\hypertarget{data}{%
\subsubsection{1.2 Data}\label{data}}

\begin{itemize}
\tightlist
\item
  This lab uses data collected from Twitter. Before the lab, we already
  collected the data and did a lot to clean it up. In the lab, will
  focus on using the data to understand social contagion.
\item
  We will look at the spread of the hashtag \#metoo on Twitter.
  \href{https://www.nytimes.com/2017/10/20/us/me-too-movement-tarana-burke.html}{The
  Me Too Campaign began in 2007}. The hashtag became extremely popular
  on Twitter starting in October, 2017. It was tweeted more than 7
  million times in the first 4 months after that.
\item
  Me Too is a social movement that calls attention to sexual assault and
  harassment. We will not be looking at the text of any of the tweets in
  this lab. For our purposes here, we just want to look at how being
  exposed to other people talking about \#metoo on Twitter influences
  someone to also talk about \#metoo on Twitter.
\end{itemize}

    \hypertarget{section-2-setup}{%
\subsection{Section 2: Setup}\label{section-2-setup}}

\hypertarget{import-python-modules}{%
\paragraph{Import Python Modules}\label{import-python-modules}}

    \begin{Verbatim}[commandchars=\\\{\}]
{\color{incolor}In [{\color{incolor}1}]:} \PY{k+kn}{import} \PY{n+nn}{pandas} \PY{k}{as} \PY{n+nn}{pd}
        \PY{k+kn}{import} \PY{n+nn}{numpy} \PY{k}{as} \PY{n+nn}{np}
        \PY{k+kn}{import} \PY{n+nn}{matplotlib}\PY{n+nn}{.}\PY{n+nn}{pyplot} \PY{k}{as} \PY{n+nn}{plt}
        \PY{k+kn}{from} \PY{n+nn}{matplotlib}\PY{n+nn}{.}\PY{n+nn}{colors} \PY{k}{import} \PY{n}{LogNorm}
        
        \PY{c+c1}{\PYZsh{}plt.style.use(\PYZsq{}fivethirtyeight\PYZsq{})}
        \PY{o}{\PYZpc{}}\PY{k}{matplotlib} inline
\end{Verbatim}


    \begin{Verbatim}[commandchars=\\\{\}]
{\color{incolor}In [{\color{incolor}2}]:} \PY{c+c1}{\PYZsh{}plt.style.use(\PYZsq{}fivethirtyeight\PYZsq{})}
        \PY{n}{plt}\PY{o}{.}\PY{n}{style}\PY{o}{.}\PY{n}{use}\PY{p}{(}\PY{l+s+s1}{\PYZsq{}}\PY{l+s+s1}{ggplot}\PY{l+s+s1}{\PYZsq{}}\PY{p}{)}
\end{Verbatim}


    \hypertarget{section-3-spread-of-metoo}{%
\subsection{Section 3: Spread of
MeToo}\label{section-3-spread-of-metoo}}

\hypertarget{load-data-helper-function}{%
\paragraph{Load data \& helper
function}\label{load-data-helper-function}}

    \begin{Verbatim}[commandchars=\\\{\}]
{\color{incolor}In [{\color{incolor}3}]:} \PY{k}{def} \PY{n+nf}{downcast}\PY{p}{(}\PY{n}{df}\PY{p}{)}\PY{p}{:}
            \PY{n}{tmp} \PY{o}{=} \PY{n}{df}\PY{o}{.}\PY{n}{select\PYZus{}dtypes}\PY{p}{(}\PY{n+nb}{int}\PY{p}{)}
            \PY{k}{for} \PY{n}{c} \PY{o+ow}{in} \PY{n}{tmp}\PY{o}{.}\PY{n}{columns}\PY{o}{.}\PY{n}{values}\PY{p}{:}
                \PY{n}{df}\PY{p}{[}\PY{n}{c}\PY{p}{]} \PY{o}{=} \PY{n}{pd}\PY{o}{.}\PY{n}{to\PYZus{}numeric}\PY{p}{(}\PY{n}{df}\PY{p}{[}\PY{n}{c}\PY{p}{]}\PY{p}{,} \PY{n}{downcast}\PY{o}{=}\PY{l+s+s1}{\PYZsq{}}\PY{l+s+s1}{unsigned}\PY{l+s+s1}{\PYZsq{}}\PY{p}{)}
            
            \PY{n}{tmp} \PY{o}{=} \PY{n}{df}\PY{o}{.}\PY{n}{select\PYZus{}dtypes}\PY{p}{(}\PY{n+nb}{float}\PY{p}{)}
            \PY{k}{for} \PY{n}{c} \PY{o+ow}{in} \PY{n}{tmp}\PY{o}{.}\PY{n}{columns}\PY{o}{.}\PY{n}{values}\PY{p}{:}
                \PY{n}{df}\PY{p}{[}\PY{n}{c}\PY{p}{]} \PY{o}{=} \PY{n}{pd}\PY{o}{.}\PY{n}{to\PYZus{}numeric}\PY{p}{(}\PY{n}{df}\PY{p}{[}\PY{n}{c}\PY{p}{]}\PY{p}{,} \PY{n}{downcast}\PY{o}{=}\PY{l+s+s1}{\PYZsq{}}\PY{l+s+s1}{float}\PY{l+s+s1}{\PYZsq{}}\PY{p}{)}
        
        \PY{n}{pcts} \PY{o}{=} \PY{n}{pd}\PY{o}{.}\PY{n}{read\PYZus{}csv}\PY{p}{(}\PY{l+s+s1}{\PYZsq{}}\PY{l+s+s1}{data/daily\PYZus{}pcts.tsv}\PY{l+s+s1}{\PYZsq{}}\PY{p}{,} \PY{n}{sep}\PY{o}{=}\PY{l+s+s1}{\PYZsq{}}\PY{l+s+se}{\PYZbs{}t}\PY{l+s+s1}{\PYZsq{}}\PY{p}{,} \PY{n}{index\PYZus{}col}\PY{o}{=}\PY{l+m+mi}{0}\PY{p}{)}
        \PY{n}{pcts}\PY{o}{.}\PY{n}{index} \PY{o}{=} \PY{n}{pd}\PY{o}{.}\PY{n}{to\PYZus{}datetime}\PY{p}{(}\PY{n}{pcts}\PY{o}{.}\PY{n}{index}\PY{p}{)}
        \PY{n}{downcast}\PY{p}{(}\PY{n}{pcts}\PY{p}{)}
        
        \PY{n}{counts} \PY{o}{=} \PY{n}{pd}\PY{o}{.}\PY{n}{read\PYZus{}csv}\PY{p}{(}\PY{l+s+s1}{\PYZsq{}}\PY{l+s+s1}{data/daily\PYZus{}counts.tsv}\PY{l+s+s1}{\PYZsq{}}\PY{p}{,} \PY{n}{sep}\PY{o}{=}\PY{l+s+s1}{\PYZsq{}}\PY{l+s+se}{\PYZbs{}t}\PY{l+s+s1}{\PYZsq{}}\PY{p}{,} \PY{n}{index\PYZus{}col}\PY{o}{=}\PY{l+m+mi}{0}\PY{p}{)}
        \PY{n}{counts}\PY{o}{.}\PY{n}{index} \PY{o}{=} \PY{n}{pd}\PY{o}{.}\PY{n}{to\PYZus{}datetime}\PY{p}{(}\PY{n}{counts}\PY{o}{.}\PY{n}{index}\PY{p}{)}
        \PY{n}{downcast}\PY{p}{(}\PY{n}{counts}\PY{p}{)}
\end{Verbatim}


    \hypertarget{every-day-a-few-people-used-the-hashtag-metoo}{%
\paragraph{Every day, a few people used the hashtag
\#metoo}\label{every-day-a-few-people-used-the-hashtag-metoo}}

\begin{itemize}
\tightlist
\item
  Take a look at the chart below. It shows how many tweets were made
  every day using the hashtag \#metoo
\item
  Notice that people on twitter did use the hashtag \#metoo before
  October 15, 2017.
\end{itemize}

    \begin{Verbatim}[commandchars=\\\{\}]
{\color{incolor}In [{\color{incolor}4}]:} \PY{n}{event} \PY{o}{=} \PY{n}{pd}\PY{o}{.}\PY{n}{datetime}\PY{p}{(}\PY{l+m+mi}{2017}\PY{p}{,}\PY{l+m+mi}{10}\PY{p}{,}\PY{l+m+mi}{15}\PY{p}{)}
        
        \PY{n}{counts}\PY{p}{[}\PY{n}{counts}\PY{o}{.}\PY{n}{index} \PY{o}{\PYZlt{}} \PY{n}{event}\PY{p}{]}\PY{o}{.}\PY{n}{tweets}\PY{o}{.}\PY{n}{plot}\PY{o}{.}\PY{n}{area}\PY{p}{(}\PY{n}{figsize}\PY{o}{=}\PY{p}{(}\PY{l+m+mi}{5}\PY{p}{,}\PY{l+m+mi}{6}\PY{p}{)}\PY{p}{,} \PY{n}{title}\PY{o}{=}\PY{l+s+s1}{\PYZsq{}}\PY{l+s+s1}{\PYZsh{}metoo Tweets per Day}\PY{l+s+s1}{\PYZsq{}}\PY{p}{)}
        \PY{n}{plt}\PY{o}{.}\PY{n}{tight\PYZus{}layout}\PY{p}{(}\PY{p}{)}
\end{Verbatim}


    \begin{center}
    \adjustimage{max size={0.9\linewidth}{0.9\paperheight}}{output_8_0.png}
    \end{center}
    { \hspace*{\fill} \\}
    
    \hypertarget{then-something-happened}{%
\paragraph{Then something happened}\label{then-something-happened}}

\begin{itemize}
\tightlist
\item
  On October 15, 2017, producer and actress Alyssa Milano used the
  hashtag, and there were 600,000 tweets about it within a single day.
\item
  Since then, the number of \#metoo tweets per day has been mugh higher.
\item
  It is possible that some of these people are coming up with the
  hashtag on their own. However, we saw above that before Milano
  tweeted, not many tweets used the hashtag. The extreme and fast growth
  of the hashtag suggests a process of social influence or contagion:
  people are getting the idea from seeing other people use it.
\end{itemize}

    \begin{Verbatim}[commandchars=\\\{\}]
{\color{incolor}In [{\color{incolor}5}]:} \PY{n}{counts}\PY{o}{.}\PY{n}{tweets}\PY{o}{.}\PY{n}{plot}\PY{o}{.}\PY{n}{area}\PY{p}{(}\PY{n}{figsize}\PY{o}{=}\PY{p}{(}\PY{l+m+mi}{10}\PY{p}{,}\PY{l+m+mi}{7}\PY{p}{)}\PY{p}{,} \PY{n}{title}\PY{o}{=}\PY{l+s+s1}{\PYZsq{}}\PY{l+s+s1}{\PYZsh{}metoo Tweets per Day}\PY{l+s+s1}{\PYZsq{}}\PY{p}{)}
        \PY{n}{plt}\PY{o}{.}\PY{n}{tight\PYZus{}layout}\PY{p}{(}\PY{p}{)}
\end{Verbatim}


    \begin{center}
    \adjustimage{max size={0.9\linewidth}{0.9\paperheight}}{output_10_0.png}
    \end{center}
    { \hspace*{\fill} \\}
    
    \hypertarget{short-answer}{%
\paragraph{Short answer:}\label{short-answer}}

\begin{itemize}
\tightlist
\item
  After the initial October 15 rush of tweets, there are several other
  days where the number of tweets jumps up. In a few sentences, write
  what you think might be happening on those days.
\end{itemize}

    \hypertarget{your-answer-here}{%
\paragraph{Your answer here:}\label{your-answer-here}}

    \hypertarget{cumulative-plots}{%
\paragraph{Cumulative plots}\label{cumulative-plots}}

\begin{itemize}
\tightlist
\item
  In the chart below, we show how the total number of tweets about
  \#metoo grows over time. This is called a ``cumulative'' plot, because
  the tweets ``accumulate'' over time. That is, the value for each day
  is the total number of tweets about \#metoo that have happened up to
  that day. As time goes on, the number of tweets that have ever been
  made can only go up, so each day is as high or higher than the day
  before it (we ignore deleted tweets).
\item
  We also show how the number of people using the hashtag grows over
  time.
\item
  Notice that participation does not grow at a constant rate. Instead,
  it grows in bursts, just like the daily tweeting happens in bursts.
\end{itemize}

    \begin{Verbatim}[commandchars=\\\{\}]
{\color{incolor}In [{\color{incolor}6}]:} \PY{n}{counts}\PY{o}{.}\PY{n}{cumsum}\PY{p}{(}\PY{p}{)}\PY{o}{.}\PY{n}{plot}\PY{p}{(}\PY{n}{figsize}\PY{o}{=}\PY{p}{(}\PY{l+m+mi}{10}\PY{p}{,}\PY{l+m+mi}{7}\PY{p}{)}\PY{p}{,} \PY{n}{title}\PY{o}{=}\PY{l+s+s1}{\PYZsq{}}\PY{l+s+s1}{All\PYZhy{}Time Total Number of \PYZsh{}metoo Tweets and People Tweeting \PYZsh{}metoo}\PY{l+s+s1}{\PYZsq{}}\PY{p}{)}
        \PY{n}{plt}\PY{o}{.}\PY{n}{tight\PYZus{}layout}\PY{p}{(}\PY{p}{)}
\end{Verbatim}


    \begin{center}
    \adjustimage{max size={0.9\linewidth}{0.9\paperheight}}{output_14_0.png}
    \end{center}
    { \hspace*{\fill} \\}
    
    \hypertarget{most-metoo-tweets-are-social}{%
\paragraph{Most \#metoo tweets are
social}\label{most-metoo-tweets-are-social}}

\begin{itemize}
\tightlist
\item
  There are different kinds of tweets on Twitter. These are the two most
  common ones:

  \begin{itemize}
  \tightlist
  \item
    \textbf{New tweets} are when someone writes a tweet and posts it.
  \item
    \textbf{Retweets} are when someone shares another person's tweet.
  \end{itemize}
\item
  Retweets involve a user interacting with other users: sharing what
  they wrote. They're a clear example of contagion because one user
  literally got the tweet from another user.
\item
  In the two charts below, we see that the majority of tweets about
  \#metoo are retweets.
\end{itemize}

    \begin{Verbatim}[commandchars=\\\{\}]
{\color{incolor}In [{\color{incolor}7}]:} \PY{n}{pcts}\PY{o}{.}\PY{n}{plot}\PY{o}{.}\PY{n}{line}\PY{p}{(}\PY{n}{y}\PY{o}{=}\PY{p}{[}\PY{l+s+s1}{\PYZsq{}}\PY{l+s+s1}{retweet}\PY{l+s+s1}{\PYZsq{}}\PY{p}{,} \PY{l+s+s1}{\PYZsq{}}\PY{l+s+s1}{new\PYZus{}tweet}\PY{l+s+s1}{\PYZsq{}}\PY{p}{]}\PY{p}{,} \PY{n}{ylim} \PY{o}{=} \PY{p}{[}\PY{l+m+mi}{0}\PY{p}{,}\PY{l+m+mi}{1}\PY{p}{]}\PY{p}{,}
                      \PY{n}{figsize}\PY{o}{=}\PY{p}{(}\PY{l+m+mi}{10}\PY{p}{,}\PY{l+m+mi}{6}\PY{p}{)}\PY{p}{,} \PY{n}{title}\PY{o}{=}\PY{l+s+s1}{\PYZsq{}}\PY{l+s+s1}{\PYZsh{}MeToo Tweets by Type}\PY{l+s+s1}{\PYZsq{}}\PY{p}{)}
        \PY{n}{plt}\PY{o}{.}\PY{n}{tight\PYZus{}layout}\PY{p}{(}\PY{p}{)}
\end{Verbatim}


    \begin{center}
    \adjustimage{max size={0.9\linewidth}{0.9\paperheight}}{output_16_0.png}
    \end{center}
    { \hspace*{\fill} \\}
    
    \hypertarget{section-4-how-much-exposure-to-metoo-do-people-have-before-joining}{%
\subsection{Section 4: How much exposure to \#metoo do people have
before
joining?}\label{section-4-how-much-exposure-to-metoo-do-people-have-before-joining}}

\begin{itemize}
\tightlist
\item
  The cell below loads and prepares data for this section. You don't
  need to understand it right now. Run the cell and scroll down.
\end{itemize}

    \begin{Verbatim}[commandchars=\\\{\}]
{\color{incolor}In [{\color{incolor}8}]:} \PY{c+c1}{\PYZsh{}load data}
        \PY{n}{tfm} \PY{o}{=} \PY{n}{pd}\PY{o}{.}\PY{n}{read\PYZus{}csv}\PY{p}{(}\PY{l+s+s1}{\PYZsq{}}\PY{l+s+s1}{data/tweets\PYZus{}friends\PYZus{}mentions\PYZus{}all.tsv}\PY{l+s+s1}{\PYZsq{}}\PY{p}{,} \PY{n}{sep}\PY{o}{=}\PY{l+s+s1}{\PYZsq{}}\PY{l+s+se}{\PYZbs{}t}\PY{l+s+s1}{\PYZsq{}}\PY{p}{)}
        \PY{c+c1}{\PYZsh{}focus on people who have seen \PYZlt{} 2,000 tweets (most people)}
        \PY{n}{tfm} \PY{o}{=} \PY{n}{tfm}\PY{p}{[}\PY{p}{(}\PY{n}{tfm}\PY{o}{.}\PY{n}{tweets\PYZus{}seen} \PY{o}{\PYZlt{}}\PY{o}{=}\PY{l+m+mi}{2000}\PY{p}{)}\PY{p}{]}
        \PY{n}{downcast}\PY{p}{(}\PY{n}{tfm}\PY{p}{)}
        
        \PY{n}{tweets\PYZus{}seen} \PY{o}{=} \PY{n}{pd}\PY{o}{.}\PY{n}{read\PYZus{}csv}\PY{p}{(}\PY{l+s+s1}{\PYZsq{}}\PY{l+s+s1}{data/tweets\PYZus{}seen.tsv}\PY{l+s+s1}{\PYZsq{}}\PY{p}{,} \PY{n}{sep}\PY{o}{=}\PY{l+s+s1}{\PYZsq{}}\PY{l+s+se}{\PYZbs{}t}\PY{l+s+s1}{\PYZsq{}}\PY{p}{)}
        \PY{n}{friends\PYZus{}seen} \PY{o}{=} \PY{n}{pd}\PY{o}{.}\PY{n}{read\PYZus{}csv}\PY{p}{(}\PY{l+s+s1}{\PYZsq{}}\PY{l+s+s1}{data/friends\PYZus{}seen.tsv}\PY{l+s+s1}{\PYZsq{}}\PY{p}{,} \PY{n}{sep}\PY{o}{=}\PY{l+s+s1}{\PYZsq{}}\PY{l+s+se}{\PYZbs{}t}\PY{l+s+s1}{\PYZsq{}}\PY{p}{)}
        
        \PY{c+c1}{\PYZsh{}helper function }
        \PY{k}{def} \PY{n+nf}{calc\PYZus{}probs}\PY{p}{(}\PY{n}{df}\PY{p}{)}\PY{p}{:}
            \PY{n}{df}\PY{p}{[}\PY{l+s+s1}{\PYZsq{}}\PY{l+s+s1}{pct\PYZus{}tweeted}\PY{l+s+s1}{\PYZsq{}}\PY{p}{]} \PY{o}{=} \PY{n}{df}\PY{o}{.}\PY{n}{n\PYZus{}tweeted} \PY{o}{/} \PY{n}{df}\PY{o}{.}\PY{n}{n\PYZus{}tweeted}\PY{o}{.}\PY{n}{max}\PY{p}{(}\PY{p}{)}
            
            \PY{k}{if} \PY{l+s+s1}{\PYZsq{}}\PY{l+s+s1}{n\PYZus{}viewers}\PY{l+s+s1}{\PYZsq{}} \PY{o+ow}{in} \PY{n}{df}\PY{o}{.}\PY{n}{columns}\PY{p}{:}
                \PY{n}{df}\PY{p}{[}\PY{l+s+s1}{\PYZsq{}}\PY{l+s+s1}{prob\PYZus{}at\PYZus{}exact\PYZus{}exposure}\PY{l+s+s1}{\PYZsq{}}\PY{p}{]} \PY{o}{=} \PY{n}{df}\PY{o}{.}\PY{n}{n\PYZus{}tweets} \PY{o}{/} \PY{p}{(}\PY{n}{df}\PY{o}{.}\PY{n}{n\PYZus{}viewers} \PY{o}{+} \PY{n}{df}\PY{o}{.}\PY{n}{n\PYZus{}tweets}\PY{p}{)} 
            \PY{k}{else}\PY{p}{:}
                \PY{n}{df}\PY{p}{[}\PY{l+s+s1}{\PYZsq{}}\PY{l+s+s1}{prob\PYZus{}at\PYZus{}exact\PYZus{}exposure}\PY{l+s+s1}{\PYZsq{}}\PY{p}{]} \PY{o}{=} \PY{n}{np}\PY{o}{.}\PY{n}{NaN}
            
            \PY{n}{df}\PY{o}{.}\PY{n}{sort\PYZus{}values}\PY{p}{(}\PY{n}{by}\PY{o}{=}\PY{l+s+s1}{\PYZsq{}}\PY{l+s+s1}{views}\PY{l+s+s1}{\PYZsq{}}\PY{p}{,} \PY{n}{ascending}\PY{o}{=}\PY{k+kc}{False}\PY{p}{,} \PY{n}{inplace}\PY{o}{=}\PY{k+kc}{True}\PY{p}{)}
            \PY{n}{df}\PY{p}{[}\PY{l+s+s1}{\PYZsq{}}\PY{l+s+s1}{n\PYZus{}viewers\PYZus{}metoo}\PY{l+s+s1}{\PYZsq{}}\PY{p}{]} \PY{o}{=} \PY{n}{df}\PY{o}{.}\PY{n}{n\PYZus{}tweets}\PY{o}{.}\PY{n}{cumsum}\PY{p}{(}\PY{p}{)}
            \PY{n}{df}\PY{o}{.}\PY{n}{sort\PYZus{}values}\PY{p}{(}\PY{n}{by}\PY{o}{=}\PY{l+s+s1}{\PYZsq{}}\PY{l+s+s1}{views}\PY{l+s+s1}{\PYZsq{}}\PY{p}{,} \PY{n}{ascending}\PY{o}{=}\PY{k+kc}{True}\PY{p}{,} \PY{n}{inplace}\PY{o}{=}\PY{k+kc}{True}\PY{p}{)}
            \PY{n}{df}\PY{p}{[}\PY{l+s+s1}{\PYZsq{}}\PY{l+s+s1}{joiners\PYZus{}prob\PYZus{}at\PYZus{}exact\PYZus{}exposure}\PY{l+s+s1}{\PYZsq{}}\PY{p}{]} \PY{o}{=} \PY{n}{df}\PY{o}{.}\PY{n}{n\PYZus{}tweets} \PY{o}{/} \PY{p}{(}\PY{n}{df}\PY{o}{.}\PY{n}{n\PYZus{}viewers\PYZus{}metoo}\PY{p}{)}     
            \PY{n}{downcast}\PY{p}{(}\PY{n}{df}\PY{p}{)}
            \PY{k}{return} \PY{n}{df}
            
        \PY{n}{tweets\PYZus{}seen} \PY{o}{=} \PY{n}{calc\PYZus{}probs}\PY{p}{(}\PY{n}{tweets\PYZus{}seen}\PY{p}{)}
        \PY{n}{friends\PYZus{}seen} \PY{o}{=} \PY{n}{calc\PYZus{}probs}\PY{p}{(}\PY{n}{friends\PYZus{}seen}\PY{p}{)}
\end{Verbatim}


    \begin{itemize}
\tightlist
\item
  We know that people are getting the idea to use \#metoo from each
  other. In this section, we'll look at how much exposure to \#metoo
  they had from other users. That is, we will see how many times people
  were exposed to \#metoo before they participated by tweeting about it
  themselves.
\item
  In the plot below, we see one dot for every person who tweeted
  \#metoo.

  \begin{itemize}
  \tightlist
  \item
    The dot indicates how many tweets (x axis) and friends (y axis) they
    saw tweeting about \#metoo before they made their first \#metoo
    tweet. So someone who saw 200 tweets from 50 friends would be shown
    as a dot at x=200, y=50
  \end{itemize}
\end{itemize}

\textbf{Note:} The charts in the rest of the lab have more code than
usual. This is just so that we can set up the labels and make the charts
pretty. Don't worry about the details of that formatting code (lines
starting with \texttt{ax} or \texttt{plt}).

    \begin{Verbatim}[commandchars=\\\{\}]
{\color{incolor}In [{\color{incolor}9}]:} \PY{n}{fig}\PY{p}{,} \PY{n}{ax} \PY{o}{=} \PY{n}{plt}\PY{o}{.}\PY{n}{subplots}\PY{p}{(}\PY{n}{figsize}\PY{o}{=}\PY{p}{(}\PY{l+m+mi}{10}\PY{p}{,}\PY{l+m+mi}{6}\PY{p}{)}\PY{p}{)}
        \PY{n}{plt}\PY{o}{.}\PY{n}{scatter}\PY{p}{(}\PY{n}{x}\PY{o}{=}\PY{n}{tfm}\PY{o}{.}\PY{n}{tweets\PYZus{}seen}\PY{p}{,} \PY{n}{y}\PY{o}{=}\PY{n}{tfm}\PY{o}{.}\PY{n}{friends\PYZus{}seen}\PY{p}{,} \PY{n}{s}\PY{o}{=}\PY{l+m+mi}{1}\PY{p}{,} \PY{n}{alpha}\PY{o}{=}\PY{o}{.}\PY{l+m+mi}{5}\PY{p}{)}
        \PY{n}{ax}\PY{o}{.}\PY{n}{set\PYZus{}ylabel}\PY{p}{(}\PY{l+s+s1}{\PYZsq{}}\PY{l+s+s1}{Friends Seen}\PY{l+s+s1}{\PYZsq{}}\PY{p}{)}
        \PY{n}{ax}\PY{o}{.}\PY{n}{set\PYZus{}xlabel}\PY{p}{(}\PY{l+s+s1}{\PYZsq{}}\PY{l+s+s1}{Tweets Seen}\PY{l+s+s1}{\PYZsq{}}\PY{p}{)}
        \PY{n}{ax}\PY{o}{.}\PY{n}{set\PYZus{}title}\PY{p}{(}\PY{l+s+s2}{\PYZdq{}}\PY{l+s+s2}{Users}\PY{l+s+s2}{\PYZsq{}}\PY{l+s+s2}{ Exposure before First \PYZsh{}metoo Tweet}\PY{l+s+s2}{\PYZdq{}}\PY{p}{)}
        \PY{n}{ax}\PY{o}{.}\PY{n}{set\PYZus{}xlim}\PY{p}{(}\PY{l+m+mi}{0}\PY{p}{,}\PY{l+m+mi}{2000}\PY{p}{)}
        \PY{n}{ax}\PY{o}{.}\PY{n}{set\PYZus{}ylim}\PY{p}{(}\PY{l+m+mi}{0}\PY{p}{,}\PY{l+m+mi}{1150}\PY{p}{)}
        \PY{n}{plt}\PY{o}{.}\PY{n}{show}\PY{p}{(}\PY{p}{)}
\end{Verbatim}


    \begin{center}
    \adjustimage{max size={0.9\linewidth}{0.9\paperheight}}{output_20_0.png}
    \end{center}
    { \hspace*{\fill} \\}
    
    \hypertarget{seeing-this-more-clearly}{%
\paragraph{Seeing this more clearly}\label{seeing-this-more-clearly}}

\begin{itemize}
\tightlist
\item
  It is hard to see individual people in the chart above, because there
  are 2 million people all crammed into a small amount of space on our
  screens.
\item
  The chart below color codes each area of the chart by how many people
  are there. This lets us see how densely packed the points are. Or, in
  other words, it lets us see what values are most and least common.

  \begin{itemize}
  \tightlist
  \item
    The dark blue areas have very few people (\(10^0 = 1\)).
  \item
    The green areas have more people (e.g. \(10^3 = 1,000\))
  \item
    The bright yellow areas have a lot of people (\(10^5 = 100,000\))
  \end{itemize}
\end{itemize}

    \begin{Verbatim}[commandchars=\\\{\}]
{\color{incolor}In [{\color{incolor}10}]:} \PY{n}{fig}\PY{p}{,} \PY{n}{ax} \PY{o}{=} \PY{n}{plt}\PY{o}{.}\PY{n}{subplots}\PY{p}{(}\PY{n}{figsize}\PY{o}{=}\PY{p}{(}\PY{l+m+mi}{12}\PY{p}{,}\PY{l+m+mi}{6}\PY{p}{)}\PY{p}{)}
         \PY{c+c1}{\PYZsh{}plt.hist2d(x=tfm.tweets\PYZus{}seen, y=tfm.friends\PYZus{}seen, bins=60, norm=LogNorm())}
         \PY{c+c1}{\PYZsh{}hexbin is another way of getting the same plot as 2dhist}
         \PY{n}{plt}\PY{o}{.}\PY{n}{hexbin}\PY{p}{(}\PY{n}{x}\PY{o}{=}\PY{n}{tfm}\PY{o}{.}\PY{n}{tweets\PYZus{}seen}\PY{p}{,} \PY{n}{y}\PY{o}{=}\PY{n}{tfm}\PY{o}{.}\PY{n}{friends\PYZus{}seen}\PY{p}{,} \PY{n}{bins}\PY{o}{=}\PY{l+m+mi}{800000}\PY{p}{,} \PY{n}{norm}\PY{o}{=}\PY{n}{LogNorm}\PY{p}{(}\PY{p}{)}\PY{p}{)}
         \PY{n}{ax}\PY{o}{.}\PY{n}{set\PYZus{}ylabel}\PY{p}{(}\PY{l+s+s1}{\PYZsq{}}\PY{l+s+s1}{Friends Seen}\PY{l+s+s1}{\PYZsq{}}\PY{p}{)}
         \PY{n}{ax}\PY{o}{.}\PY{n}{set\PYZus{}xlabel}\PY{p}{(}\PY{l+s+s1}{\PYZsq{}}\PY{l+s+s1}{Tweets Seen}\PY{l+s+s1}{\PYZsq{}}\PY{p}{)}
         \PY{n}{ax}\PY{o}{.}\PY{n}{set\PYZus{}title}\PY{p}{(}\PY{l+s+s2}{\PYZdq{}}\PY{l+s+s2}{Users}\PY{l+s+s2}{\PYZsq{}}\PY{l+s+s2}{ Exposure before First \PYZsh{}metoo Tweet}\PY{l+s+s2}{\PYZdq{}}\PY{p}{)}
         \PY{n}{ax}\PY{o}{.}\PY{n}{set\PYZus{}xlim}\PY{p}{(}\PY{l+m+mi}{0}\PY{p}{,}\PY{l+m+mi}{2000}\PY{p}{)}
         \PY{n}{ax}\PY{o}{.}\PY{n}{set\PYZus{}ylim}\PY{p}{(}\PY{l+m+mi}{0}\PY{p}{,}\PY{l+m+mi}{1150}\PY{p}{)}
         \PY{n}{ax}\PY{o}{.}\PY{n}{grid}\PY{p}{(}\PY{k+kc}{True}\PY{p}{)}
         \PY{n}{plt}\PY{o}{.}\PY{n}{colorbar}\PY{p}{(}\PY{p}{)}
         \PY{n}{plt}\PY{o}{.}\PY{n}{show}\PY{p}{(}\PY{p}{)}
\end{Verbatim}


    \begin{center}
    \adjustimage{max size={0.9\linewidth}{0.9\paperheight}}{output_22_0.png}
    \end{center}
    { \hspace*{\fill} \\}
    
    \hypertarget{how-much-exposure-do-most-people-have-when-they-join}{%
\paragraph{How much exposure do most people have when they
join?}\label{how-much-exposure-do-most-people-have-when-they-join}}

\begin{itemize}
\tightlist
\item
  The charts above tell us that joining is most common at low levels of
  exposure, but it is hard to see in them how much exposure is required
  for a majority of people.
\item
  In the next chart, we look at the same data with a cumulative plot. We
  show the cumulative number of people who have joined by the time their
  exposure reaches x.

  \begin{itemize}
  \tightlist
  \item
    So, for example, the chart below says that of the people who
    eventually tweeted \#metoo, 80\% of them made their first tweet
    after seeing fewer than 100 of the people they follow tweet \#metoo.
  \end{itemize}
\item
  \textbf{Try it:} run the code below several times, picking different
  numbers for the \texttt{maximum\_exposure} in order to zoom in and
  out.
\end{itemize}

    \begin{Verbatim}[commandchars=\\\{\}]
{\color{incolor}In [{\color{incolor}11}]:} \PY{n}{maximum\PYZus{}exposure} \PY{o}{=} \PY{l+m+mi}{500}
         
         \PY{n}{fig}\PY{p}{,} \PY{n}{ax} \PY{o}{=} \PY{n}{plt}\PY{o}{.}\PY{n}{subplots}\PY{p}{(}\PY{n}{figsize}\PY{o}{=}\PY{p}{(}\PY{l+m+mi}{10}\PY{p}{,}\PY{l+m+mi}{8}\PY{p}{)}\PY{p}{)}
         \PY{n}{plt}\PY{o}{.}\PY{n}{plot}\PY{p}{(}\PY{n}{friends\PYZus{}seen}\PY{o}{.}\PY{n}{views}\PY{p}{,} \PY{n}{friends\PYZus{}seen}\PY{o}{.}\PY{n}{pct\PYZus{}tweeted}\PY{p}{,} \PY{n}{label}\PY{o}{=}\PY{l+s+s1}{\PYZsq{}}\PY{l+s+s1}{Friends Seen}\PY{l+s+s1}{\PYZsq{}}\PY{p}{)}
         \PY{n}{plt}\PY{o}{.}\PY{n}{plot}\PY{p}{(}\PY{n}{tweets\PYZus{}seen}\PY{o}{.}\PY{n}{views}\PY{p}{,} \PY{n}{tweets\PYZus{}seen}\PY{o}{.}\PY{n}{pct\PYZus{}tweeted}\PY{p}{,} \PY{n}{label}\PY{o}{=}\PY{l+s+s1}{\PYZsq{}}\PY{l+s+s1}{Tweets Seen}\PY{l+s+s1}{\PYZsq{}}\PY{p}{,} \PY{n}{color}\PY{o}{=}\PY{l+s+s1}{\PYZsq{}}\PY{l+s+s1}{black}\PY{l+s+s1}{\PYZsq{}}\PY{p}{)}
         \PY{n}{ax}\PY{o}{.}\PY{n}{set\PYZus{}ylabel}\PY{p}{(}\PY{l+s+s1}{\PYZsq{}}\PY{l+s+s1}{Fraction Who Joined \PYZsh{}metoo by Exposure X}\PY{l+s+s1}{\PYZsq{}}\PY{p}{)}
         \PY{n}{ax}\PY{o}{.}\PY{n}{set\PYZus{}xlabel}\PY{p}{(}\PY{l+s+s1}{\PYZsq{}}\PY{l+s+s1}{Exposure}\PY{l+s+s1}{\PYZsq{}}\PY{p}{)}
         \PY{n}{ax}\PY{o}{.}\PY{n}{set\PYZus{}xlim}\PY{p}{(}\PY{o}{\PYZhy{}}\PY{l+m+mi}{1}\PY{p}{,} \PY{n}{maximum\PYZus{}exposure}\PY{p}{)}
         \PY{n}{ax}\PY{o}{.}\PY{n}{set\PYZus{}ylim}\PY{p}{(}\PY{l+m+mi}{0}\PY{p}{,} \PY{l+m+mi}{1}\PY{p}{)}
         \PY{n}{ax}\PY{o}{.}\PY{n}{set\PYZus{}title}\PY{p}{(}\PY{l+s+s1}{\PYZsq{}}\PY{l+s+s1}{Fraction of Joiners who join by the Time Their Exposure is X}\PY{l+s+s1}{\PYZsq{}}\PY{p}{)}
         \PY{n}{plt}\PY{o}{.}\PY{n}{legend}\PY{p}{(}\PY{n}{loc}\PY{o}{=}\PY{l+s+s1}{\PYZsq{}}\PY{l+s+s1}{lower right}\PY{l+s+s1}{\PYZsq{}}\PY{p}{)}
         \PY{n}{plt}\PY{o}{.}\PY{n}{tight\PYZus{}layout}\PY{p}{(}\PY{p}{)}
         \PY{n}{plt}\PY{o}{.}\PY{n}{show}\PY{p}{(}\PY{p}{)}
\end{Verbatim}


    \begin{center}
    \adjustimage{max size={0.9\linewidth}{0.9\paperheight}}{output_24_0.png}
    \end{center}
    { \hspace*{\fill} \\}
    
    \hypertarget{short-answers}{%
\paragraph{Short Answers}\label{short-answers}}

\begin{itemize}
\tightlist
\item
  The lines in the chart don't start at 0 on the Y axis; they start at
  6.3\%. In a few sentences, explain what you think this means.
\item
  The majority of people (51\%) have tweeted by the time they get to
  what level of exposure? Write how many tweets they have seen, and how
  many friends they have seen tweeting in a full sentence.
\end{itemize}

    Your answers here

    \hypertarget{section-5-does-more-exposure-make-people-less-likely-to-join}{%
\subsection{\texorpdfstring{Section 5: Does more exposure make people
\emph{less} likely to
join?}{Section 5: Does more exposure make people less likely to join?}}\label{section-5-does-more-exposure-make-people-less-likely-to-join}}

\begin{itemize}
\tightlist
\item
  We saw above that most people join at low levels of exposure.
\item
  Does this mean that more exposure makes people less likely to join?We
  will find out in this section
\end{itemize}

\hypertarget{short-answer}{%
\paragraph{Short answer}\label{short-answer}}

\begin{itemize}
\tightlist
\item
  Before we do, make a prediction: Do you think people who have been
  exposed to \#metoo many times are more or less likely to tweet about
  it than those who have been exposed a few times? Explain why in a few
  sentences.
\end{itemize}

    Your Answer Here

    \hypertarget{lets-see-if-youre-right}{%
\paragraph{Let's see if you're right!}\label{lets-see-if-youre-right}}

\hypertarget{probability-of-tweeting-metoo-based-on-amount-of-exposure}{%
\paragraph{Probability of tweeting \#metoo based on amount of
exposure}\label{probability-of-tweeting-metoo-based-on-amount-of-exposure}}

\begin{itemize}
\tightlist
\item
  The next two charts show how many times someone has been exposed to
  the hashtag \#metoo on the X axis.

  \begin{itemize}
  \tightlist
  \item
    The number of tweets they have seen is the number of times people
    they follow have tweeted about \#metoo
  \item
    The number of people seen is the number of people they follow who
    have tweeted about \#metoo.
  \end{itemize}
\item
  On the Y axis, they show the probability that someone will make their
  first \#metoo tweet, after having seen X number of tweets or people
  tweeting about \#metoo.

  \begin{itemize}
  \tightlist
  \item
    Probability ranges from 0 (it definitely will not happen) to 1 (it
    definitely will happen). Getting ``heads'' on a coin toss has a
    probability of 0.5 (50\%). Rolling a one on a six-sided die has a
    probability of \(1/6 \approx 0.1667\) (16.67\%).
  \item
    So, for example, the first chart shows that people who have seen
    exactly 20 of the people they follow tweet about \#metoo have a 0.02
    probability of making their first \#metoo tweet. That's a 2\% chance
    of ``catching'' the ``contagion.''
  \end{itemize}
\item
  \textbf{Try it:} change the \texttt{max\_exposure} variable to zoom in
  and out on the x axis.
\end{itemize}

    \begin{Verbatim}[commandchars=\\\{\}]
{\color{incolor}In [{\color{incolor}12}]:} \PY{n}{max\PYZus{}exposure} \PY{o}{=} \PY{l+m+mi}{500}
         
         \PY{n}{fig}\PY{p}{,} \PY{n}{ax} \PY{o}{=} \PY{n}{plt}\PY{o}{.}\PY{n}{subplots}\PY{p}{(}\PY{n}{figsize}\PY{o}{=}\PY{p}{(}\PY{l+m+mi}{10}\PY{p}{,}\PY{l+m+mi}{8}\PY{p}{)}\PY{p}{)}
         \PY{n}{plt}\PY{o}{.}\PY{n}{scatter}\PY{p}{(}\PY{n}{x}\PY{o}{=}\PY{n}{friends\PYZus{}seen}\PY{o}{.}\PY{n}{views}\PY{p}{,} \PY{n}{y}\PY{o}{=}\PY{n}{friends\PYZus{}seen}\PY{o}{.}\PY{n}{prob\PYZus{}at\PYZus{}exact\PYZus{}exposure}\PY{p}{,} \PY{n}{label}\PY{o}{=}\PY{l+s+s1}{\PYZsq{}}\PY{l+s+s1}{People Seen}\PY{l+s+s1}{\PYZsq{}}\PY{p}{)}
         \PY{n}{plt}\PY{o}{.}\PY{n}{scatter}\PY{p}{(}\PY{n}{x}\PY{o}{=}\PY{n}{tweets\PYZus{}seen}\PY{o}{.}\PY{n}{views}\PY{p}{,} \PY{n}{y}\PY{o}{=}\PY{n}{tweets\PYZus{}seen}\PY{o}{.}\PY{n}{prob\PYZus{}at\PYZus{}exact\PYZus{}exposure}\PY{p}{,} \PY{n}{label}\PY{o}{=}\PY{l+s+s1}{\PYZsq{}}\PY{l+s+s1}{Tweets Seen}\PY{l+s+s1}{\PYZsq{}}\PY{p}{,} \PY{n}{color}\PY{o}{=}\PY{l+s+s1}{\PYZsq{}}\PY{l+s+s1}{black}\PY{l+s+s1}{\PYZsq{}}\PY{p}{)}
         \PY{n}{ax}\PY{o}{.}\PY{n}{set\PYZus{}ylabel}\PY{p}{(}\PY{l+s+s1}{\PYZsq{}}\PY{l+s+s1}{Probability of Tweeting \PYZsh{}metoo for the first time}\PY{l+s+s1}{\PYZsq{}}\PY{p}{)}
         \PY{n}{ax}\PY{o}{.}\PY{n}{set\PYZus{}xlabel}\PY{p}{(}\PY{l+s+s1}{\PYZsq{}}\PY{l+s+s1}{Exposure}\PY{l+s+s1}{\PYZsq{}}\PY{p}{)}
         \PY{n}{ax}\PY{o}{.}\PY{n}{set\PYZus{}xlim}\PY{p}{(}\PY{l+m+mi}{0}\PY{p}{,} \PY{n}{max\PYZus{}exposure}\PY{p}{)}
         \PY{n}{ax}\PY{o}{.}\PY{n}{set\PYZus{}ylim}\PY{p}{(}\PY{l+m+mi}{0}\PY{p}{,} \PY{l+m+mf}{0.3}\PY{p}{)}
         \PY{n}{ax}\PY{o}{.}\PY{n}{set\PYZus{}title}\PY{p}{(}\PY{l+s+s1}{\PYZsq{}}\PY{l+s+s1}{Fraction of People with Exact Exposure X Who Tweet}\PY{l+s+s1}{\PYZsq{}}\PY{p}{)}
         \PY{n}{plt}\PY{o}{.}\PY{n}{legend}\PY{p}{(}\PY{n}{loc}\PY{o}{=}\PY{l+s+s1}{\PYZsq{}}\PY{l+s+s1}{upper left}\PY{l+s+s1}{\PYZsq{}}\PY{p}{)}
         \PY{n}{plt}\PY{o}{.}\PY{n}{tight\PYZus{}layout}\PY{p}{(}\PY{p}{)}
         \PY{n}{plt}\PY{o}{.}\PY{n}{show}\PY{p}{(}\PY{p}{)}
\end{Verbatim}


    \begin{center}
    \adjustimage{max size={0.9\linewidth}{0.9\paperheight}}{output_30_0.png}
    \end{center}
    { \hspace*{\fill} \\}
    
    \hypertarget{short-answer}{%
\paragraph{Short Answer}\label{short-answer}}

\begin{itemize}
\tightlist
\item
  We see that people with more exposure to \#metoo are more likely to
  tweet about it than those with less exposure. Was your prediction
  above correct?
\end{itemize}

    Your answer here

    \hypertarget{why-does-this-happen}{%
\subsubsection{Why does this happen?}\label{why-does-this-happen}}

\begin{itemize}
\item
  The chart above tells us that high exposure people are more likely to
  tweet than low exposure people. But the charts before that told us
  that most people who tweet about \#metoo do so at lower levels of
  exposure.
\item
  This happens because there are different numbers of people at each
  level of exposure. 41 million people saw one tweet about \#metoo, but
  only 12 million people saw 5 tweets about it, and only 164 thousand
  people saw 100 tweets about it. You can see this in the chart below.
\item
  Thus, even though only 1 out of 1,000 people who see \#metoo one time
  tweet about it, so many people see it one time that the number of
  people tweeting after one exposure is 41,865. People who see \#metoo
  100 times are 12 times more likely to tweet about it than those who
  only saw it once (12 out of 1,000). But because so few people are
  exposed 100 times, the total number tweeting after 100 exposures is
  just 1,977 (that is 20 times fewer people)!

  \((0.001*41865986) > (0.012*164788)\)

  \(41865 > 1977\)
\end{itemize}

    \begin{Verbatim}[commandchars=\\\{\}]
{\color{incolor}In [{\color{incolor}13}]:} \PY{n}{max\PYZus{}exposure} \PY{o}{=} \PY{l+m+mi}{100}
         
         \PY{n}{fig}\PY{p}{,} \PY{n}{ax} \PY{o}{=} \PY{n}{plt}\PY{o}{.}\PY{n}{subplots}\PY{p}{(}\PY{n}{figsize}\PY{o}{=}\PY{p}{(}\PY{l+m+mi}{10}\PY{p}{,}\PY{l+m+mi}{8}\PY{p}{)}\PY{p}{)}
         \PY{c+c1}{\PYZsh{}plt.scatter(x=friends\PYZus{}seen.views, y=friends\PYZus{}seen.n\PYZus{}viewers/1000000, label=\PYZsq{}People\PYZsq{})}
         \PY{n}{plt}\PY{o}{.}\PY{n}{scatter}\PY{p}{(}\PY{n}{x}\PY{o}{=}\PY{n}{tweets\PYZus{}seen}\PY{o}{.}\PY{n}{views}\PY{p}{,} \PY{n}{y}\PY{o}{=}\PY{n}{tweets\PYZus{}seen}\PY{o}{.}\PY{n}{n\PYZus{}viewers}\PY{o}{/}\PY{l+m+mi}{1000000}\PY{p}{,} \PY{n}{label}\PY{o}{=}\PY{l+s+s1}{\PYZsq{}}\PY{l+s+s1}{Tweets}\PY{l+s+s1}{\PYZsq{}}\PY{p}{,} \PY{n}{color}\PY{o}{=}\PY{l+s+s1}{\PYZsq{}}\PY{l+s+s1}{black}\PY{l+s+s1}{\PYZsq{}}\PY{p}{)}
         \PY{n}{ax}\PY{o}{.}\PY{n}{set\PYZus{}ylabel}\PY{p}{(}\PY{l+s+s1}{\PYZsq{}}\PY{l+s+s1}{Millions of People}\PY{l+s+s1}{\PYZsq{}}\PY{p}{)}
         \PY{n}{ax}\PY{o}{.}\PY{n}{set\PYZus{}xlabel}\PY{p}{(}\PY{l+s+s1}{\PYZsq{}}\PY{l+s+s1}{Exposure}\PY{l+s+s1}{\PYZsq{}}\PY{p}{)}
         \PY{n}{ax}\PY{o}{.}\PY{n}{set\PYZus{}xlim}\PY{p}{(}\PY{l+m+mi}{0}\PY{p}{,} \PY{n}{max\PYZus{}exposure}\PY{p}{)}
         \PY{n}{ax}\PY{o}{.}\PY{n}{set\PYZus{}ylim}\PY{p}{(}\PY{l+m+mi}{0}\PY{p}{,} \PY{l+m+mi}{43}\PY{p}{)}
         \PY{n}{ax}\PY{o}{.}\PY{n}{set\PYZus{}title}\PY{p}{(}\PY{l+s+s1}{\PYZsq{}}\PY{l+s+s1}{Number of People with Exact Exposure X}\PY{l+s+s1}{\PYZsq{}}\PY{p}{)}
         \PY{n}{plt}\PY{o}{.}\PY{n}{legend}\PY{p}{(}\PY{n}{loc}\PY{o}{=}\PY{l+s+s1}{\PYZsq{}}\PY{l+s+s1}{upper right}\PY{l+s+s1}{\PYZsq{}}\PY{p}{)}
         \PY{n}{plt}\PY{o}{.}\PY{n}{tight\PYZus{}layout}\PY{p}{(}\PY{p}{)}
         \PY{n}{plt}\PY{o}{.}\PY{n}{show}\PY{p}{(}\PY{p}{)}
\end{Verbatim}


    \begin{center}
    \adjustimage{max size={0.9\linewidth}{0.9\paperheight}}{output_34_0.png}
    \end{center}
    { \hspace*{\fill} \\}
    
    \hypertarget{short-answers}{%
\paragraph{Short Answers}\label{short-answers}}

\begin{itemize}
\tightlist
\item
  As people's exposure to \#metoo increases, so does their probability
  of tweeting \#metoo. In other words, sometimes people need multiple
  exposures to ``catch'' \#metoo.

  \begin{itemize}
  \tightlist
  \item
    Assuming people see 100\% of tweets in their feed, does this mean
    that \#metoo is a complex contagion? Write a few sentences
    explaining your answer.
  \item
    What if people do not see all the tweets in their feed? Does your
    answer change? Write a few sentences explaining why or why not.
  \item
    What about the people who tweet about \#metoo before any of their
    friends do? How do they fit in?
  \end{itemize}
\end{itemize}

    \hypertarget{your-answer-here}{%
\paragraph{Your answer here}\label{your-answer-here}}

    \hypertarget{section-6-are-close-friends-more-influential-than-other-friends}{%
\subsection{Section 6: Are close friends more influential than other
friends?}\label{section-6-are-close-friends-more-influential-than-other-friends}}

\begin{itemize}
\item
  \begin{itemize}
  \tightlist
  \item
    Before, we were looking at the users that people follow. Following
    shows interest and means that you'll see what that person tweets in
    your feed. On twitter, people you follow are called your friends.

    \begin{itemize}
    \tightlist
    \item
      But some users follow a lot of people, and they might not care
      about all of them equally.
    \item
      A stronger signal that one user is paying attention to another
      user is a \texttt{mention}: when someone tags someone else in a
      tweet. Users generally mention fewer people than they follow, and
      they might mention someone they do not follow.
    \item
      For this lab, we will compare what happens when we look at all
      friends with when we look only at friends that a user mentions by
      name.
    \end{itemize}
  \end{itemize}
\item
  Below, we will compare the effects of seeing someone you follow tweet
  with seeing a friend you have mentioned tweet.
\end{itemize}

\hypertarget{load-and-process-data}{%
\paragraph{Load and process data}\label{load-and-process-data}}

\begin{itemize}
\tightlist
\item
  Run this code and scroll down. Don't worry about how it works for now.
\end{itemize}

    \begin{Verbatim}[commandchars=\\\{\}]
{\color{incolor}In [{\color{incolor}14}]:} \PY{n}{close\PYZus{}friends\PYZus{}seen} \PY{o}{=} \PY{n}{pd}\PY{o}{.}\PY{n}{read\PYZus{}csv}\PY{p}{(}\PY{l+s+s1}{\PYZsq{}}\PY{l+s+s1}{data/mentions\PYZus{}friends\PYZus{}seen.tsv}\PY{l+s+s1}{\PYZsq{}}\PY{p}{,} \PY{n}{sep}\PY{o}{=}\PY{l+s+s1}{\PYZsq{}}\PY{l+s+se}{\PYZbs{}t}\PY{l+s+s1}{\PYZsq{}}\PY{p}{)}
         \PY{n}{far\PYZus{}friends\PYZus{}seen} \PY{o}{=} \PY{n}{pd}\PY{o}{.}\PY{n}{read\PYZus{}csv}\PY{p}{(}\PY{l+s+s1}{\PYZsq{}}\PY{l+s+s1}{data/far\PYZus{}friends\PYZus{}seen.tsv}\PY{l+s+s1}{\PYZsq{}}\PY{p}{,} \PY{n}{sep}\PY{o}{=}\PY{l+s+s1}{\PYZsq{}}\PY{l+s+se}{\PYZbs{}t}\PY{l+s+s1}{\PYZsq{}}\PY{p}{)}
         
         \PY{n}{close\PYZus{}friends\PYZus{}seen} \PY{o}{=} \PY{n}{calc\PYZus{}probs}\PY{p}{(}\PY{n}{close\PYZus{}friends\PYZus{}seen}\PY{p}{)}
         \PY{n}{far\PYZus{}friends\PYZus{}seen} \PY{o}{=} \PY{n}{calc\PYZus{}probs}\PY{p}{(}\PY{n}{far\PYZus{}friends\PYZus{}seen}\PY{p}{)}
\end{Verbatim}


    \hypertarget{the-influence-of-close-friends}{%
\paragraph{The influence of close
friends}\label{the-influence-of-close-friends}}

\begin{itemize}
\tightlist
\item
  The plot below shows the cumulative number of people who have tweeted
  \#metoo by the time they were exposed by X number of friends or close
  friends.
\item
  \textbf{Try it:} You can zoom in and out in this chart by changing the
  value of \texttt{max\_exposure}.
\item
  We see that almost half tweet before any of their close friends do,
  and around 10\% of people tweet before any of their not-close friends
  do.
\item
  We also see that people who have seen a small number of their close
  friends tweet are much more likely to tweet than those who have seen
  the same number of not-close friends tweet.
\end{itemize}

    \begin{Verbatim}[commandchars=\\\{\}]
{\color{incolor}In [{\color{incolor}15}]:} \PY{n}{max\PYZus{}exposure} \PY{o}{=} \PY{l+m+mi}{300}
         
         \PY{n}{fig}\PY{p}{,} \PY{n}{ax} \PY{o}{=} \PY{n}{plt}\PY{o}{.}\PY{n}{subplots}\PY{p}{(}\PY{n}{figsize}\PY{o}{=}\PY{p}{(}\PY{l+m+mi}{10}\PY{p}{,}\PY{l+m+mi}{8}\PY{p}{)}\PY{p}{)}
         \PY{n}{plt}\PY{o}{.}\PY{n}{plot}\PY{p}{(}\PY{n}{far\PYZus{}friends\PYZus{}seen}\PY{o}{.}\PY{n}{views}\PY{p}{,} \PY{n}{far\PYZus{}friends\PYZus{}seen}\PY{o}{.}\PY{n}{pct\PYZus{}tweeted}\PY{p}{,} \PY{n}{label}\PY{o}{=}\PY{l+s+s1}{\PYZsq{}}\PY{l+s+s1}{Not\PYZhy{}close friends Seen}\PY{l+s+s1}{\PYZsq{}}\PY{p}{)}
         \PY{n}{plt}\PY{o}{.}\PY{n}{plot}\PY{p}{(}\PY{n}{close\PYZus{}friends\PYZus{}seen}\PY{o}{.}\PY{n}{views}\PY{p}{,} \PY{n}{close\PYZus{}friends\PYZus{}seen}\PY{o}{.}\PY{n}{pct\PYZus{}tweeted}\PY{p}{,} \PY{n}{label}\PY{o}{=}\PY{l+s+s1}{\PYZsq{}}\PY{l+s+s1}{Close friends Seen}\PY{l+s+s1}{\PYZsq{}}\PY{p}{)}
         \PY{n}{ax}\PY{o}{.}\PY{n}{set\PYZus{}ylabel}\PY{p}{(}\PY{l+s+s1}{\PYZsq{}}\PY{l+s+s1}{Fraction Who Joined \PYZsh{}metoo}\PY{l+s+s1}{\PYZsq{}}\PY{p}{)}
         \PY{n}{ax}\PY{o}{.}\PY{n}{set\PYZus{}xlabel}\PY{p}{(}\PY{l+s+s1}{\PYZsq{}}\PY{l+s+s1}{Exposure}\PY{l+s+s1}{\PYZsq{}}\PY{p}{)}
         \PY{n}{ax}\PY{o}{.}\PY{n}{set\PYZus{}xlim}\PY{p}{(}\PY{o}{\PYZhy{}}\PY{l+m+mi}{2}\PY{p}{,} \PY{n}{max\PYZus{}exposure}\PY{p}{)}
         \PY{n}{ax}\PY{o}{.}\PY{n}{set\PYZus{}ylim}\PY{p}{(}\PY{l+m+mi}{0}\PY{p}{,} \PY{l+m+mf}{1.01}\PY{p}{)}
         \PY{n}{ax}\PY{o}{.}\PY{n}{set\PYZus{}title}\PY{p}{(}\PY{l+s+s1}{\PYZsq{}}\PY{l+s+s1}{Fraction of Joiners who join up to Exact Exposure X}\PY{l+s+s1}{\PYZsq{}}\PY{p}{)}
         \PY{n}{plt}\PY{o}{.}\PY{n}{legend}\PY{p}{(}\PY{n}{loc}\PY{o}{=}\PY{l+s+s1}{\PYZsq{}}\PY{l+s+s1}{lower right}\PY{l+s+s1}{\PYZsq{}}\PY{p}{)}
         \PY{n}{plt}\PY{o}{.}\PY{n}{show}\PY{p}{(}\PY{p}{)}
\end{Verbatim}


    \begin{center}
    \adjustimage{max size={0.9\linewidth}{0.9\paperheight}}{output_40_0.png}
    \end{center}
    { \hspace*{\fill} \\}
    
    \hypertarget{section-7-whats-the-right-number-of-friends}{%
\subsection{Section 7: What's the right number of
friends?}\label{section-7-whats-the-right-number-of-friends}}

\begin{itemize}
\tightlist
\item
  We have seen so far that seeing tweets from more friends makes people
  more likely to join, but is this just because more friends tweeting
  means they have seen more tweets?
\item
  To answer this question, we will compare people who have seen the
  exact same number of tweets, but from different numbers of friends.
  For example, if I have seen 20 tweets about \#metoo, they might all
  come from one person who tweeted 20 times, from 20 different people
  who tweeted once, from 10 people who each tweeted twice, etc.
\end{itemize}

\textbf{Helper functions:} The cell below loads and prepares some data.
Run it and scroll down to continue the lab.

    \begin{Verbatim}[commandchars=\\\{\}]
{\color{incolor}In [{\color{incolor}16}]:} \PY{n}{vot} \PY{o}{=} \PY{n}{pd}\PY{o}{.}\PY{n}{read\PYZus{}csv}\PY{p}{(}\PY{l+s+s1}{\PYZsq{}}\PY{l+s+s1}{data/views\PYZus{}over\PYZus{}time\PYZus{}grouped.tsv}\PY{l+s+s1}{\PYZsq{}}\PY{p}{,} \PY{n}{sep}\PY{o}{=}\PY{l+s+s1}{\PYZsq{}}\PY{l+s+se}{\PYZbs{}t}\PY{l+s+s1}{\PYZsq{}}\PY{p}{)}
         \PY{n}{first\PYZus{}new} \PY{o}{=} \PY{n}{pd}\PY{o}{.}\PY{n}{read\PYZus{}csv}\PY{p}{(}\PY{l+s+s1}{\PYZsq{}}\PY{l+s+s1}{data/tweets\PYZus{}friends\PYZus{}mentions\PYZus{}new.tsv}\PY{l+s+s1}{\PYZsq{}}\PY{p}{,} \PY{n}{sep}\PY{o}{=}\PY{l+s+s1}{\PYZsq{}}\PY{l+s+se}{\PYZbs{}t}\PY{l+s+s1}{\PYZsq{}}\PY{p}{)}
         \PY{n}{first\PYZus{}rt} \PY{o}{=} \PY{n}{pd}\PY{o}{.}\PY{n}{read\PYZus{}csv}\PY{p}{(}\PY{l+s+s1}{\PYZsq{}}\PY{l+s+s1}{data/tweets\PYZus{}friends\PYZus{}mentions\PYZus{}rt.tsv}\PY{l+s+s1}{\PYZsq{}}\PY{p}{,} \PY{n}{sep}\PY{o}{=}\PY{l+s+s1}{\PYZsq{}}\PY{l+s+se}{\PYZbs{}t}\PY{l+s+s1}{\PYZsq{}}\PY{p}{)}
         
         \PY{k}{def} \PY{n+nf}{prep\PYZus{}views}\PY{p}{(}\PY{n}{tweets}\PY{p}{,} \PY{n}{vot}\PY{p}{)}\PY{p}{:}
             \PY{n}{tweets}\PY{p}{[}\PY{l+s+s1}{\PYZsq{}}\PY{l+s+s1}{action}\PY{l+s+s1}{\PYZsq{}}\PY{p}{]} \PY{o}{=} \PY{l+m+mi}{1}
             \PY{n}{grouped} \PY{o}{=} \PY{n}{tweets}\PY{o}{.}\PY{n}{groupby}\PY{p}{(}\PY{p}{[}\PY{l+s+s1}{\PYZsq{}}\PY{l+s+s1}{tweets\PYZus{}seen}\PY{l+s+s1}{\PYZsq{}}\PY{p}{,} \PY{l+s+s1}{\PYZsq{}}\PY{l+s+s1}{friends\PYZus{}seen}\PY{l+s+s1}{\PYZsq{}}\PY{p}{]}\PY{p}{)}\PY{o}{.}\PY{n}{sum}\PY{p}{(}\PY{p}{)}\PY{o}{.}\PY{n}{reset\PYZus{}index}\PY{p}{(}\PY{p}{)}
             \PY{n}{df} \PY{o}{=} \PY{n}{vot}\PY{o}{.}\PY{n}{merge}\PY{p}{(}\PY{n}{grouped}\PY{p}{,} \PY{n}{on}\PY{o}{=}\PY{p}{[}\PY{l+s+s1}{\PYZsq{}}\PY{l+s+s1}{tweets\PYZus{}seen}\PY{l+s+s1}{\PYZsq{}}\PY{p}{,} \PY{l+s+s1}{\PYZsq{}}\PY{l+s+s1}{friends\PYZus{}seen}\PY{l+s+s1}{\PYZsq{}}\PY{p}{]}\PY{p}{,} \PY{n}{how}\PY{o}{=}\PY{l+s+s1}{\PYZsq{}}\PY{l+s+s1}{left}\PY{l+s+s1}{\PYZsq{}}\PY{p}{)}
             \PY{n}{df}\PY{p}{[}\PY{l+s+s1}{\PYZsq{}}\PY{l+s+s1}{pct}\PY{l+s+s1}{\PYZsq{}}\PY{p}{]} \PY{o}{=} \PY{p}{(}\PY{n}{df}\PY{o}{.}\PY{n}{action} \PY{o}{/} \PY{n}{df}\PY{o}{.}\PY{n}{n}\PY{p}{)} \PY{o}{*} \PY{l+m+mi}{100}
             \PY{n}{downcast}\PY{p}{(}\PY{n}{df}\PY{p}{)}
             \PY{k}{return} \PY{n}{df}
         
         \PY{n}{first\PYZus{}tweet} \PY{o}{=} \PY{n}{prep\PYZus{}views}\PY{p}{(}\PY{n}{tfm}\PY{p}{,} \PY{n}{vot}\PY{p}{)}
         \PY{n}{first\PYZus{}new} \PY{o}{=} \PY{n}{prep\PYZus{}views}\PY{p}{(}\PY{n}{first\PYZus{}new}\PY{p}{,} \PY{n}{vot}\PY{p}{)}
         \PY{n}{first\PYZus{}rt} \PY{o}{=} \PY{n}{prep\PYZus{}views}\PY{p}{(}\PY{n}{first\PYZus{}rt}\PY{p}{,} \PY{n}{vot}\PY{p}{)}
\end{Verbatim}


    \hypertarget{the-right-number-of-friends}{%
\paragraph{The right number of
friends}\label{the-right-number-of-friends}}

\begin{itemize}
\tightlist
\item
  \textbf{Pick a number:} Decide what level of exposure you want to look
  at, for example people who have seen 10, 20, or 50 tweets about
  \#metoo. Set this number in the code as \texttt{n\_tweets}. You can
  pick any number from 5 to 100. Try it several times with different
  numbers.
\item
  The first graph shows you ``of the people who saw \texttt{n\_tweets},
  how many saw those tweets all from 1 friend, 2 friends, etc.?''
\item
  The second graph shows you the probability that someone will tweet
  \#metoo, if they have seen \texttt{n\_tweets} about it from X
  different friends.
\end{itemize}

    \begin{Verbatim}[commandchars=\\\{\}]
{\color{incolor}In [{\color{incolor}17}]:} \PY{n}{n\PYZus{}tweets} \PY{o}{=} \PY{l+m+mi}{20}
         
         \PY{n}{df} \PY{o}{=} \PY{n}{first\PYZus{}tweet}\PY{p}{[}\PY{n}{first\PYZus{}tweet}\PY{o}{.}\PY{n}{tweets\PYZus{}seen} \PY{o}{==} \PY{n}{n\PYZus{}tweets}\PY{p}{]}
         
         \PY{n}{fig}\PY{p}{,} \PY{n}{ax} \PY{o}{=} \PY{n}{plt}\PY{o}{.}\PY{n}{subplots}\PY{p}{(}\PY{n}{figsize}\PY{o}{=}\PY{p}{(}\PY{l+m+mi}{10}\PY{p}{,}\PY{l+m+mi}{8}\PY{p}{)}\PY{p}{)}
         \PY{n}{plt}\PY{o}{.}\PY{n}{bar}\PY{p}{(}\PY{n}{x}\PY{o}{=}\PY{n}{df}\PY{o}{.}\PY{n}{friends\PYZus{}seen}\PY{p}{,} \PY{n}{height}\PY{o}{=}\PY{n}{df}\PY{o}{.}\PY{n}{n}\PY{p}{)}
         \PY{n}{ax}\PY{o}{.}\PY{n}{set\PYZus{}ylabel}\PY{p}{(}\PY{l+s+s1}{\PYZsq{}}\PY{l+s+s1}{Number of People}\PY{l+s+s1}{\PYZsq{}}\PY{p}{)}
         \PY{n}{ax}\PY{o}{.}\PY{n}{set\PYZus{}xlabel}\PY{p}{(}\PY{l+s+s1}{\PYZsq{}}\PY{l+s+s1}{Number of people the tweets came from}\PY{l+s+s1}{\PYZsq{}}\PY{p}{)}
         \PY{n}{ax}\PY{o}{.}\PY{n}{set\PYZus{}title}\PY{p}{(}\PY{l+s+s1}{\PYZsq{}}\PY{l+s+s1}{Number of people who have seen }\PY{l+s+s1}{\PYZsq{}} \PY{o}{+} \PY{n+nb}{str}\PY{p}{(}\PY{n}{n\PYZus{}tweets}\PY{p}{)} \PY{o}{+} \PY{l+s+s1}{\PYZsq{}}\PY{l+s+s1}{ tweets from X different people}\PY{l+s+s1}{\PYZsq{}}\PY{p}{)}
         \PY{n}{plt}\PY{o}{.}\PY{n}{tight\PYZus{}layout}\PY{p}{(}\PY{p}{)}
         \PY{n}{plt}\PY{o}{.}\PY{n}{show}\PY{p}{(}\PY{p}{)}
         
         \PY{n}{fig}\PY{p}{,} \PY{n}{ax} \PY{o}{=} \PY{n}{plt}\PY{o}{.}\PY{n}{subplots}\PY{p}{(}\PY{n}{figsize}\PY{o}{=}\PY{p}{(}\PY{l+m+mi}{10}\PY{p}{,}\PY{l+m+mi}{8}\PY{p}{)}\PY{p}{)}
         \PY{n}{plt}\PY{o}{.}\PY{n}{bar}\PY{p}{(}\PY{n}{x}\PY{o}{=}\PY{n}{df}\PY{o}{.}\PY{n}{friends\PYZus{}seen}\PY{p}{,} \PY{n}{height}\PY{o}{=}\PY{n}{df}\PY{o}{.}\PY{n}{pct}\PY{p}{)}
         \PY{n}{ax}\PY{o}{.}\PY{n}{set\PYZus{}ylabel}\PY{p}{(}\PY{l+s+s1}{\PYZsq{}}\PY{l+s+s1}{\PYZpc{}}\PY{l+s+s1}{ who tweeted \PYZsh{}metoo}\PY{l+s+s1}{\PYZsq{}}\PY{p}{)}
         \PY{n}{ax}\PY{o}{.}\PY{n}{set\PYZus{}xlabel}\PY{p}{(}\PY{l+s+s1}{\PYZsq{}}\PY{l+s+s1}{Number of people the tweets came from}\PY{l+s+s1}{\PYZsq{}}\PY{p}{)}
         \PY{n}{ax}\PY{o}{.}\PY{n}{set\PYZus{}title}\PY{p}{(}\PY{l+s+s1}{\PYZsq{}}\PY{l+s+s1}{Probability of tweeting \PYZsh{}metoo after seeing }\PY{l+s+s1}{\PYZsq{}} \PY{o}{+} \PY{n+nb}{str}\PY{p}{(}\PY{n}{n\PYZus{}tweets}\PY{p}{)} \PY{o}{+} \PY{l+s+s1}{\PYZsq{}}\PY{l+s+s1}{ tweets from X different people}\PY{l+s+s1}{\PYZsq{}}\PY{p}{)}
         \PY{n}{plt}\PY{o}{.}\PY{n}{tight\PYZus{}layout}\PY{p}{(}\PY{p}{)}
         \PY{n}{plt}\PY{o}{.}\PY{n}{show}\PY{p}{(}\PY{p}{)}
\end{Verbatim}


    \begin{center}
    \adjustimage{max size={0.9\linewidth}{0.9\paperheight}}{output_44_0.png}
    \end{center}
    { \hspace*{\fill} \\}
    
    \begin{center}
    \adjustimage{max size={0.9\linewidth}{0.9\paperheight}}{output_44_1.png}
    \end{center}
    { \hspace*{\fill} \\}
    
    \hypertarget{short-answer}{%
\paragraph{Short answer:}\label{short-answer}}

\begin{itemize}
\tightlist
\item
  It looks like people are less likely to tweet if all the tweets they
  saw came from different people (on the right in the graphs) than if
  they came from a few people (on the left side).
\item
  It also looks like people are less likely to tweet if all of the
  tweets they see come from the same person.
\item
  In a few sentences, say what you think might be causing these results.
\end{itemize}

    Your answer here

    \hypertarget{section-8-which-tweets-require-the-least-exposure}{%
\subsection{Section 8: Which tweets require the least
exposure?}\label{section-8-which-tweets-require-the-least-exposure}}

\begin{itemize}
\tightlist
\item
  Up to now, we have just looked at whether people tweeted about \#metoo
  at all. But remember from the beginning that there are different kinds
  of tweets, including new tweets and retweets. Maybe one kind of tweet
  requires less exposure than another? Here are two hypotheses:

  \begin{itemize}
  \tightlist
  \item
    Retweets require more exposure to make than new tweets, because
    someone has to see a tweet they link before they can retweet it, but
    they can write a new tweet without seeing any tweets they like.
  \item
    New tweets require more exposure, because writing something takes
    more effort and committment than pressing a button to share
    something someone else already wrote.
  \end{itemize}
\end{itemize}

\hypertarget{short-answer}{%
\paragraph{Short answer:}\label{short-answer}}

\begin{itemize}
\tightlist
\item
  In a few sentences, write which hypothesis you think is correct and
  why. (If you don't like either, you can also write your own, but you
  still have to explain why.)
\end{itemize}

    Your answer here

    \hypertarget{lets-see}{%
\paragraph{Let's see!}\label{lets-see}}

\begin{itemize}
\tightlist
\item
  The first cell loads and prepares data
\item
  Change \texttt{max\_exposure} in the second cell to zoom in and out on
  the plot
\end{itemize}

    \begin{Verbatim}[commandchars=\\\{\}]
{\color{incolor}In [{\color{incolor}18}]:} \PY{n}{ex\PYZus{}new} \PY{o}{=} \PY{n}{pd}\PY{o}{.}\PY{n}{read\PYZus{}csv}\PY{p}{(}\PY{l+s+s1}{\PYZsq{}}\PY{l+s+s1}{data/tweets\PYZus{}seen\PYZus{}before\PYZus{}new.tsv}\PY{l+s+s1}{\PYZsq{}}\PY{p}{,} \PY{n}{sep}\PY{o}{=}\PY{l+s+s1}{\PYZsq{}}\PY{l+s+se}{\PYZbs{}t}\PY{l+s+s1}{\PYZsq{}}\PY{p}{)}
         \PY{n}{ex\PYZus{}rt} \PY{o}{=} \PY{n}{pd}\PY{o}{.}\PY{n}{read\PYZus{}csv}\PY{p}{(}\PY{l+s+s1}{\PYZsq{}}\PY{l+s+s1}{data/tweets\PYZus{}seen\PYZus{}before\PYZus{}rt.tsv}\PY{l+s+s1}{\PYZsq{}}\PY{p}{,} \PY{n}{sep}\PY{o}{=}\PY{l+s+s1}{\PYZsq{}}\PY{l+s+se}{\PYZbs{}t}\PY{l+s+s1}{\PYZsq{}}\PY{p}{)}
         
         \PY{n}{ex\PYZus{}new} \PY{o}{=} \PY{n}{calc\PYZus{}probs}\PY{p}{(}\PY{n}{ex\PYZus{}new}\PY{p}{)}
         \PY{n}{ex\PYZus{}rt} \PY{o}{=} \PY{n}{calc\PYZus{}probs}\PY{p}{(}\PY{n}{ex\PYZus{}rt}\PY{p}{)}
\end{Verbatim}


    \begin{Verbatim}[commandchars=\\\{\}]
{\color{incolor}In [{\color{incolor}19}]:} \PY{n}{max\PYZus{}exposure} \PY{o}{=} \PY{l+m+mi}{500}
         
         \PY{n}{fig}\PY{p}{,} \PY{n}{ax} \PY{o}{=} \PY{n}{plt}\PY{o}{.}\PY{n}{subplots}\PY{p}{(}\PY{n}{figsize}\PY{o}{=}\PY{p}{(}\PY{l+m+mi}{10}\PY{p}{,}\PY{l+m+mi}{8}\PY{p}{)}\PY{p}{)}
         \PY{n}{plt}\PY{o}{.}\PY{n}{scatter}\PY{p}{(}\PY{n}{x}\PY{o}{=}\PY{n}{ex\PYZus{}rt}\PY{o}{.}\PY{n}{views}\PY{p}{,} \PY{n}{y}\PY{o}{=}\PY{n}{ex\PYZus{}rt}\PY{o}{.}\PY{n}{prob\PYZus{}at\PYZus{}exact\PYZus{}exposure}\PY{p}{,} \PY{n}{label}\PY{o}{=}\PY{l+s+s1}{\PYZsq{}}\PY{l+s+s1}{Retweets}\PY{l+s+s1}{\PYZsq{}}\PY{p}{)}
         \PY{n}{plt}\PY{o}{.}\PY{n}{scatter}\PY{p}{(}\PY{n}{x}\PY{o}{=}\PY{n}{ex\PYZus{}new}\PY{o}{.}\PY{n}{views}\PY{p}{,} \PY{n}{y}\PY{o}{=}\PY{n}{ex\PYZus{}new}\PY{o}{.}\PY{n}{prob\PYZus{}at\PYZus{}exact\PYZus{}exposure}\PY{p}{,} \PY{n}{label}\PY{o}{=}\PY{l+s+s1}{\PYZsq{}}\PY{l+s+s1}{New tweets}\PY{l+s+s1}{\PYZsq{}}\PY{p}{)}
         \PY{n}{ax}\PY{o}{.}\PY{n}{set\PYZus{}ylabel}\PY{p}{(}\PY{l+s+s1}{\PYZsq{}}\PY{l+s+s1}{Fraction who Tweet}\PY{l+s+s1}{\PYZsq{}}\PY{p}{)}
         \PY{n}{ax}\PY{o}{.}\PY{n}{set\PYZus{}xlabel}\PY{p}{(}\PY{l+s+s1}{\PYZsq{}}\PY{l+s+s1}{Exposure}\PY{l+s+s1}{\PYZsq{}}\PY{p}{)}
         \PY{n}{ax}\PY{o}{.}\PY{n}{set\PYZus{}xlim}\PY{p}{(}\PY{l+m+mi}{0}\PY{p}{,} \PY{n}{max\PYZus{}exposure}\PY{p}{)}
         \PY{n}{ax}\PY{o}{.}\PY{n}{set\PYZus{}ylim}\PY{p}{(}\PY{l+m+mi}{0}\PY{p}{,} \PY{l+m+mf}{0.04}\PY{p}{)}
         \PY{n}{ax}\PY{o}{.}\PY{n}{set\PYZus{}title}\PY{p}{(}\PY{l+s+s1}{\PYZsq{}}\PY{l+s+s1}{Probability of Making Retweet vs New Tweet about \PYZsh{}metoo}\PY{l+s+s1}{\PYZsq{}}\PY{p}{)}
         \PY{n}{plt}\PY{o}{.}\PY{n}{legend}\PY{p}{(}\PY{n}{loc}\PY{o}{=}\PY{l+s+s1}{\PYZsq{}}\PY{l+s+s1}{upper left}\PY{l+s+s1}{\PYZsq{}}\PY{p}{)}
         \PY{n}{plt}\PY{o}{.}\PY{n}{tight\PYZus{}layout}\PY{p}{(}\PY{p}{)}
         \PY{n}{plt}\PY{o}{.}\PY{n}{show}\PY{p}{(}\PY{p}{)}
\end{Verbatim}


    \begin{center}
    \adjustimage{max size={0.9\linewidth}{0.9\paperheight}}{output_51_0.png}
    \end{center}
    { \hspace*{\fill} \\}
    
    \hypertarget{short-answer}{%
\paragraph{Short answer:}\label{short-answer}}

\begin{itemize}
\tightlist
\item
  Does the data support your hypothesis? In a few sentences, say what
  your hypothesis was and why the data do or don't support it.
\end{itemize}

    Your answer here

    \hypertarget{review}{%
\subsection{Review}\label{review}}

\begin{itemize}
\tightlist
\item
  In the Agent Based Modeling lab, we looked at a simple simulation of
  social influence and contagion. We saw things spread among people in a
  network.
\item
  In this lab, we showed some of the complexity that happens in the real
  world by looking more closely at social contagion in data from the
  \#metoo hashtag on twitter.
\item
  We have explored different aspects of exposure and found several
  things:

  \begin{itemize}
  \tightlist
  \item
    Most people who tweet about \#metoo do so with just a little
    exposure to it
  \item
    The more exposure someone has, the more likely they are to tweet
  \item
    Close friends are more influential than other friends
  \end{itemize}
\item
  We also looked at different kinds of participation when we compared
  retweeting with making new tweets, and found that some ways of joining
  a movement require less social influence than others.
\end{itemize}


    % Add a bibliography block to the postdoc
    
    
    
    \end{document}
